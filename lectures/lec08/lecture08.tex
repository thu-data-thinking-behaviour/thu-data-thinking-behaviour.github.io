\documentclass[xetex,mathserif,serif,aspectratio=169]{beamer}

\usepackage{xltxtra}
\usepackage{color}
\usepackage{url}
\usepackage{listings}
\usepackage{fontspec}
\usepackage{geometry}
\usepackage{lastpage}
\usepackage{fancyhdr}
\usepackage{amsmath}
\usepackage{amsthm}
\usepackage{amssymb}
\usepackage{blkarray}
\usepackage{multicol}
\usepackage{relsize}
\usepackage{listings}
\usepackage{xunicode}
\usepackage{xltxtra}
\usepackage{color}
\usepackage{url}
\usefonttheme[onlymath]{serif}

\definecolor{solarized@base03}{HTML}{002B36}
\definecolor{solarized@base02}{HTML}{073642}
\definecolor{solarized@base01}{HTML}{586e75}
\definecolor{solarized@base00}{HTML}{657b83}
\definecolor{solarized@base0}{HTML}{839496}
\definecolor{solarized@base1}{HTML}{93a1a1}
\definecolor{solarized@base2}{HTML}{EEE8D5}
\definecolor{solarized@base3}{HTML}{FDF6E3}
\definecolor{solarized@yellow}{HTML}{B58900}
\definecolor{solarized@orange}{HTML}{CB4B16}
\definecolor{solarized@red}{HTML}{DC322F}
\definecolor{solarized@magenta}{HTML}{D33682}
\definecolor{solarized@violet}{HTML}{6C71C4}
\definecolor{solarized@blue}{HTML}{268BD2}
\definecolor{solarized@cyan}{HTML}{2AA198}
\definecolor{solarized@green}{HTML}{859900}
\definecolor{yaleblue}{HTML}{0E4C92}

\newcommand{\yellow}[1]{\textcolor{solarized@yellow}{#1}}
\newcommand{\orange}[1]{\textcolor{solarized@orange}{#1}}
\newcommand{\red}[1]{\textcolor{solarized@red}{#1}}
\newcommand{\magenta}[1]{\textcolor{solarized@magenta}{#1}}
\newcommand{\violet}[1]{\textcolor{solarized@violet}{#1}}
\newcommand{\blue}[1]{\textcolor{solarized@blue}{#1}}
\newcommand{\cyan}[1]{\textcolor{solarized@cyan}{#1}}
\newcommand{\green}[1]{\textcolor{solarized@green}{#1}}
\newcommand{\yblue}[1]{\textcolor{yaleblue}{#1}}
\newcommand{\base}[1]{\textcolor{solarized@base01}{#1}}


\defaultfontfeatures{Mapping=tex-text}
\hypersetup{pdfstartview={FitH}}

\newcommand{\old}[1]{\fontspec[Alternate=1,Ligatures={Common}]{Hoefler Text}\fontsize{18pt}{30pt}\selectfont #1}%
\newcommand{\oldA}[1]{\fontspec[Alternate=1,Ligatures={Common, Rare}]{Hoefler Text}\fontsize{12pt}{15pt}\selectfont #1}%
\newcommand{\oldB}[1]{\fontspec[Ligatures={Common}]{Didot}\fontsize{12pt}{15pt}\color{solarized@base02}\selectfont #1}%
\newcommand{\tfont}[1]{\fontspec[Alternate=1,Ligatures={Common}]{Hoefler Text}\fontsize{12pt}{20pt}\selectfont #1}%
\newcommand{\dfont}[1]{\fontspec[Ligatures={Common}]{Didot}\fontsize{12pt}{12pt}\selectfont #1}%

\setbeamerfont{title}{family=\old}
\setbeamerfont{author}{family=\tfont}%
\setbeamerfont{frametitle}{family=\oldA}
\setbeamerfont{date}{family=\dfont}

\setbeamertemplate{navigation symbols}{}
\setbeamertemplate{footline}[text line]{%
  \parbox{0.99\linewidth}{
    \normalsize\vspace*{-24pt}\hfill{\color{solarized@base00}\insertframenumber/\inserttotalframenumber}
  }
}


\setlength{\parindent}{0pt}
\setlength{\parskip}{12pt}

\setbeamercolor{structure}{bg=solarized@base3, fg=solarized@base02}
\setbeamercolor{titlelike}{fg=solarized@cyan}
\setbeamercolor{title}{fg=solarized@blue}
\setbeamercolor{subtitle}{fg=solarized@magenta}
\setbeamercolor{alerted text}{fg=orange}
\setbeamercolor{itemize}{fg=solarized@base02}
\setbeamercolor{background canvas}{bg=solarized@base3}
\setbeamercolor{enumerate subitem}{fg=solarized@base02}

\newcommand{\minimize}{\mathop{\mathrm{minimize}}}
\newcommand{\argmin}{\mathop{\mathrm{arg\,min}}}
\newcommand{\argmax}{\mathop{\mathrm{arg\,max}}}
\newcommand{\st}{\mathop{\mathrm{subject\,\,to}}}


\usepackage[]{algorithm2e}
\usepackage{../kbordermatrix}

\begin{document}

%%%%%%%%%%%%%%%%%%%%%%%%%%%%%%%%%%%%%%%%%%%%%%%%%%%
\begin{frame}[fragile] \frametitle{} \oldB \small

\vfill

{\fontsize{0.7cm}{0cm}\selectfont Lecture 08 \\\vspace{0.2cm} Decision Trees II}\\\vspace{0.5cm}
15 February 2016

\vspace{2cm}

\begin{minipage}{0.6\textwidth}
Taylor B. Arnold \\
Yale Statistics \\
STAT 365/665
\end{minipage}
\hfill
\begin{minipage}{0.3\textwidth}\raggedleft
\includegraphics[scale=0.3]{../yale-logo.png}
\end{minipage}%

\end{frame}

%%%%%%%%%%%%%%%%%%%%%%%%%%%%%%%%%%%%%%%%%%%%%%%%%%%
\begin{frame}[fragile] \frametitle{} \oldB \small

Updated office hours:
\begin{itemize}
\item Taylor Arnold -- Wednesdays, 16:00 - 17:00, HH 24
\item Elena Khusainova -- Tuesdays, 13:00-15:00, HH 24, Basement
\item Yu Lu -- Wednesdays, 19:00-20:30, HH 24, Basement
\item Jason Klusowski -- Thursdays, 19:00-20:30, HH 24
\end{itemize}

\end{frame}

%%%%%%%%%%%%%%%%%%%%%%%%%%%%%%%%%%%%%%%%%%%%%%%%%%%
\begin{frame}[fragile] \frametitle{} \oldB \small

Problem Set 2:
\begin{itemize}
\item We will give you full credit either way you approach this, but your
solution for Question \#4 will be much more interesting if you weight the
MSE of the census tracts by their population
\end{itemize}

\end{frame}

%%%%%%%%%%%%%%%%%%%%%%%%%%%%%%%%%%%%%%%%%%%%%%%%%%%
\begin{frame}[fragile] \frametitle{} \oldB \small

Problem Set 1:
\begin{itemize}
\item Finished grading them; will likely post grades to ClassesV2 by tomorrow morning \pause
\item Actual grades were okay; median of 9/10, mean of 8.5/10 \pause
\item However, many people not following directions:
\begin{itemize}
\item not putting files in a `.zip' directory
\item using a different zip format (.rar, .7z, .tar.gz)
\item naming files incorrectly
\item including column names, quotes, or both in the set of predicted values \pause
\end{itemize}
\item As well, many of the R and Python scripts did not actually run correctly;
a common problem being that they threw errors when they could not find global variables
or local files
\end{itemize}

\end{frame}

%%%%%%%%%%%%%%%%%%%%%%%%%%%%%%%%%%%%%%%%%%%%%%%%%%%
\begin{frame}[fragile] \frametitle{} \oldB \small

Problem Set 1, cont.
\begin{itemize}
\item Paying attention to these details is very important!
\item In almost all cases, you got full credit for the implementation and
data analysis questions as long as you wrote \textit{something} and your code
ran without any errors
\item Future problem sets will likely have fewer `parts' to them,
but the actually content will be graded more strictly, so you do not
want to lose points on silly things
\item Test your code!
\item Remember:
\begin{itemize}
\item No late submissions (unless you have a true emergency, and for undergraduates
a Deans excuse)
\item We do drop the lowest grade from the final assignment
\end{itemize}
\end{itemize}

\end{frame}


%%%%%%%%%%%%%%%%%%%%%%%%%%%%%%%%%%%%%%%%%%%%%%%%%%%
\begin{frame}[fragile] \frametitle{} \oldB \small

\begin{itemize}
\item Today:
\begin{itemize}
\item Finish decision trees
\item Walk through an analysis of trees for classification and regression
\item Introduction to SVMs (time permitting)
\end{itemize}
\end{itemize}

\end{frame}

%%%%%%%%%%%%%%%%%%%%%%%%%%%%%%%%%%%%%%%%%%%%%%%%%%%
\begin{frame}[fragile] \frametitle{} \oldB \small

\textbf{\yblue{Gradient boosted trees}}

Last time, we looked at decision trees and random forests.
There is one additional variant on these ideas which has
been shown to produce extremely powerful predictions. Called
\blue{gradient boosted trees}.


\end{frame}

%%%%%%%%%%%%%%%%%%%%%%%%%%%%%%%%%%%%%%%%%%%%%%%%%%%
\begin{frame}[fragile] \frametitle{} \oldB \small

\textbf{\yblue{Gradient boosted trees, cont.}}

The algorithm for this can conceptually be described as follows:
\begin{enumerate}
\item Fit a decision tree $T^{(1)}$ on a randomly sampled subset of the data
(but using all of the variables at each node, unlike RF) and call the
predicted values from this tree on the entire dataset $\widehat{y}^{(1)}$. \pause
\item Calculate the residuals $r^{(1)} = y - \widehat{y}^{(1)}$. \pause
\item Fit a new decision tree $T^{(2)}$ on a random resampling of the residuals
$r^{(1)}$, and calculate the predicted values $\widehat{r}^{(1)}$ for the
entire dataset. \pause
\item Now, set $\widehat{y}^{(2)}$ to be $\widehat{y}^{(1)} + \widehat{r}^{(1)}$ and
set the new residuals to be $r^{(2)} = y - \widehat{y}^{(2)}$. \pause
\item Calculate decision tree $T^{(3)}$ on a randomly sampled set of the residuals $r^{(2)}$,
and calculate the predicted values $\widehat{r}^{(2)}$ for the entire dataset. \pause
\item Repeat as needed until you have predictions $\widehat{y}^{(K)}$.
\end{enumerate}

\end{frame}

%%%%%%%%%%%%%%%%%%%%%%%%%%%%%%%%%%%%%%%%%%%%%%%%%%%
\begin{frame}[fragile] \frametitle{} \oldB \small

\textbf{\yblue{Gradient boosted trees, cont.}}

The actual algorithm is modified it two ways:
\begin{enumerate}
\item The predicted values in the terminal nodes of the decision tree $T^{(j)}$
for any $j$ are not determined just by the sampled set of data, but rather chosen
to minimize the actual loss of using the $\widehat{y}^{(j)}$.
\item The values for the predicted residuals $\widehat{r}^{(j)}$ is `shrunk' towards
zero by a factor of $0 < \rho \leq 1$. This value is called the \magenta{learning rate}.
\end{enumerate}
Due to the learning rate factor, it makes sense to center the input variables before
applying this algorithm (this is usually done by the software).

\end{frame}

\end{document}











