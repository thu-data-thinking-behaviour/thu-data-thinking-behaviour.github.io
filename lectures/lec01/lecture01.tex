\documentclass[xetex,mathserif,serif,aspectratio=169]{beamer}

\usepackage{xltxtra}
\usepackage{color}
\usepackage{url}
\usepackage{listings}
\usepackage{fontspec}
\usepackage{geometry}
\usepackage{lastpage}
\usepackage{fancyhdr}
\usepackage{amsmath}
\usepackage{amsthm}
\usepackage{amssymb}
\usepackage{blkarray}
\usepackage{multicol}
\usepackage{relsize}
\usepackage{listings}
\usepackage{xunicode}
\usepackage{xltxtra}
\usepackage{color}
\usepackage{url}
\usefonttheme[onlymath]{serif}

\definecolor{solarized@base03}{HTML}{002B36}
\definecolor{solarized@base02}{HTML}{073642}
\definecolor{solarized@base01}{HTML}{586e75}
\definecolor{solarized@base00}{HTML}{657b83}
\definecolor{solarized@base0}{HTML}{839496}
\definecolor{solarized@base1}{HTML}{93a1a1}
\definecolor{solarized@base2}{HTML}{EEE8D5}
\definecolor{solarized@base3}{HTML}{FDF6E3}
\definecolor{solarized@yellow}{HTML}{B58900}
\definecolor{solarized@orange}{HTML}{CB4B16}
\definecolor{solarized@red}{HTML}{DC322F}
\definecolor{solarized@magenta}{HTML}{D33682}
\definecolor{solarized@violet}{HTML}{6C71C4}
\definecolor{solarized@blue}{HTML}{268BD2}
\definecolor{solarized@cyan}{HTML}{2AA198}
\definecolor{solarized@green}{HTML}{859900}
\definecolor{yaleblue}{HTML}{0E4C92}

\newcommand{\yellow}[1]{\textcolor{solarized@yellow}{#1}}
\newcommand{\orange}[1]{\textcolor{solarized@orange}{#1}}
\newcommand{\red}[1]{\textcolor{solarized@red}{#1}}
\newcommand{\magenta}[1]{\textcolor{solarized@magenta}{#1}}
\newcommand{\violet}[1]{\textcolor{solarized@violet}{#1}}
\newcommand{\blue}[1]{\textcolor{solarized@blue}{#1}}
\newcommand{\cyan}[1]{\textcolor{solarized@cyan}{#1}}
\newcommand{\green}[1]{\textcolor{solarized@green}{#1}}
\newcommand{\yblue}[1]{\textcolor{yaleblue}{#1}}
\newcommand{\base}[1]{\textcolor{solarized@base01}{#1}}


\defaultfontfeatures{Mapping=tex-text}
\hypersetup{pdfstartview={FitH}}

\newcommand{\old}[1]{\fontspec[Alternate=1,Ligatures={Common}]{Hoefler Text}\fontsize{18pt}{30pt}\selectfont #1}%
\newcommand{\oldA}[1]{\fontspec[Alternate=1,Ligatures={Common, Rare}]{Hoefler Text}\fontsize{12pt}{15pt}\selectfont #1}%
\newcommand{\oldB}[1]{\fontspec[Ligatures={Common}]{Didot}\fontsize{12pt}{15pt}\color{solarized@base02}\selectfont #1}%
\newcommand{\tfont}[1]{\fontspec[Alternate=1,Ligatures={Common}]{Hoefler Text}\fontsize{12pt}{20pt}\selectfont #1}%
\newcommand{\dfont}[1]{\fontspec[Ligatures={Common}]{Didot}\fontsize{12pt}{12pt}\selectfont #1}%

\setbeamerfont{title}{family=\old}
\setbeamerfont{author}{family=\tfont}%
\setbeamerfont{frametitle}{family=\oldA}
\setbeamerfont{date}{family=\dfont}

\setbeamertemplate{navigation symbols}{}
\setbeamertemplate{footline}[text line]{%
  \parbox{0.99\linewidth}{
    \normalsize\vspace*{-24pt}\hfill{\color{solarized@base00}\insertframenumber/\inserttotalframenumber}
  }
}


\setlength{\parindent}{0pt}
\setlength{\parskip}{12pt}

\setbeamercolor{structure}{bg=solarized@base3, fg=solarized@base02}
\setbeamercolor{titlelike}{fg=solarized@cyan}
\setbeamercolor{title}{fg=solarized@blue}
\setbeamercolor{subtitle}{fg=solarized@magenta}
\setbeamercolor{alerted text}{fg=orange}
\setbeamercolor{itemize}{fg=solarized@base02}
\setbeamercolor{background canvas}{bg=solarized@base3}
\setbeamercolor{enumerate subitem}{fg=solarized@base02}

\newcommand{\minimize}{\mathop{\mathrm{minimize}}}
\newcommand{\argmin}{\mathop{\mathrm{arg\,min}}}
\newcommand{\argmax}{\mathop{\mathrm{arg\,max}}}
\newcommand{\st}{\mathop{\mathrm{subject\,\,to}}}



\title{Analyzing large-scale data: Tipping behavior in the NYC taxi dataset}
\date{2016-01-04}

\begin{document}

%%%%%%%%%%%%%%%%%%%%%%%%%%%%%%%%%%%%%%%%%%%%%%%%%%%
\begin{frame}[fragile] \frametitle{} \oldB \small

\vfill

{\fontsize{0.7cm}{0cm}\selectfont Lecture 01 \\\vspace{0.2cm} Course overview}\\\vspace{0.5cm}
20 January 2016

\vspace{2cm}

\begin{minipage}{0.6\textwidth}
Taylor B. Arnold \\
Yale Statistics \\
STAT 365/665
\end{minipage}
\hfill
\begin{minipage}{0.3\textwidth}\raggedleft
\includegraphics[scale=0.3]{../yale-logo.png}
\end{minipage}%

\end{frame}

%%%%%%%%%%%%%%%%%%%%%%%%%%%%%%%%%%%%%%%%%%%%%%%%%%%
\begin{frame}[fragile] \frametitle{} \oldB \small

\textbf{STAT 365/665: Data Mining and Machine Learning} \vspace{12pt}

Techniques for data mining and machine learning are covered from both a statistical and a computational perspective, including support vector machines, bagging, boosting, neural networks, and other nonlinear and nonparametric regression methods. The course gives the basic ideas and intuition behind these methods, a more formal understanding of how and why they work, and opportunities to experiment with machine-learning algorithms and apply them to data.

\end{frame}

%%%%%%%%%%%%%%%%%%%%%%%%%%%%%%%%%%%%%%%%%%%%%%%%%%%
\begin{frame}[fragile] \frametitle{} \oldB \small

I would not say that description is inaccurate, but it is incredibly vague.
I will concentrate in particular on:

\begin{itemize}
\item computation aspects; both theory and practical considerations
\item machine learning as applied data analysis
\end{itemize}

This will be quite different than the way the course was taught in recent years.

\end{frame}

%%%%%%%%%%%%%%%%%%%%%%%%%%%%%%%%%%%%%%%%%%%%%%%%%%%
\begin{frame}[fragile] \frametitle{} \oldB \small

Classes will typically consists of about 30 minutes of lecture followed by
interactive coding, simulations, and data analysis.

\end{frame}

%%%%%%%%%%%%%%%%%%%%%%%%%%%%%%%%%%%%%%%%%%%%%%%%%%%
\begin{frame}[fragile] \frametitle{} \oldB \small

{\bf Suggested Prerequisites:}
\begin{itemize}\setlength\itemsep{0em}
\item Introductory statistical theory
\item Exposure to applied data analysis
\item Familiar with R; proficient with R or Python
\end{itemize}

\end{frame}

%%%%%%%%%%%%%%%%%%%%%%%%%%%%%%%%%%%%%%%%%%%%%%%%%%%
\begin{frame}[fragile] \frametitle{} \oldB \small

{\bf A rough outline of the course:}
\begin{itemize}\setlength\itemsep{0em}
\item 3 weeks - linear smoothers and support vector machines
\item 3 weeks - introduction to neural networks
\item 4 weeks - applications to computer vision
\item 3 weeks - applications to natural language processing
\end{itemize}

\end{frame}

%%%%%%%%%%%%%%%%%%%%%%%%%%%%%%%%%%%%%%%%%%%%%%%%%%%
\begin{frame}[fragile] \frametitle{} \oldB \small

{\bf References:}

\begin{itemize}\setlength\itemsep{0em}
\item Ian Goodfellow, Aaron Courville and Yoshua Bengio. \textit{Deep Learning}. Book in preparation for MIT Press. \url{http://www.deeplearningbook.org/}.
\item Jerome Friedman, Trevor Hastie and Robert Tibshirani. \textit{The Elements of Statistical Learning}. Springer, Berlin: Springer Series in Statistics, 2011.
\item Cosma Rohilla Shalizi. \textit{Advanced Data Analysis from an Elementary Point of View}. Book in preparation. \url{http://www.stat.cmu.edu/~cshalizi/ADAfaEPoV/}.
\end{itemize}

\end{frame}

%%%%%%%%%%%%%%%%%%%%%%%%%%%%%%%%%%%%%%%%%%%%%%%%%%%
\begin{frame}[fragile] \frametitle{} \oldB \small

{\bf Some datasets that we will look at include:}

\begin{itemize}\setlength\itemsep{0em}
\item Taxi Data: \url{http://www.nyc.gov/html/tlc/html/about/trip_record_data.shtml}
\item Million Song Dataset: \url{http://labrosa.ee.columbia.edu/millionsong/}
\item MNIST: \url{http://yann.lecun.com/exdb/mnist/}
\item CIFAR-10/CIFAR-100: \url{https://www.cs.toronto.edu/~kriz/cifar.html}
\item The Street View House Numbers (SVHN) Dataset: \url{http://ufldl.stanford.edu/housenumbers/}
\item ILSVRC: \url{http://image-net.org/challenges/LSVRC/2015/}
\item Microsoft Common Images in Context (MS COCO): \url{http://mscoco.org/dataset/}
\item Wikipedia
\end{itemize}

\end{frame}

%%%%%%%%%%%%%%%%%%%%%%%%%%%%%%%%%%%%%%%%%%%%%%%%%%%
\begin{frame}[fragile] \frametitle{} \oldB \small

{\bf Course website:}

A copy of the whole course syllabus, including a more detailed description
of topics I plan to cover are on the course website.

\begin{center}
\url{http://www.stat.yale.edu/~tba3/stat665/}
\end{center}

This is where all lecture notes, homeworks, and other references will appear.

\end{frame}

%%%%%%%%%%%%%%%%%%%%%%%%%%%%%%%%%%%%%%%%%%%%%%%%%%%
\begin{frame}[fragile] \frametitle{} \oldB \small

{\bf TA's:}

Yu Lu, Jason Klusowski

Office hour times, formats, locations to be determined.

\end{frame}

%%%%%%%%%%%%%%%%%%%%%%%%%%%%%%%%%%%%%%%%%%%%%%%%%%%
\begin{frame}[fragile] \frametitle{}  \oldB \small

{\bf STAT 365 vs. STAT 665}

Same requirements and assignments; however grading schemes may be
different, even week to week.

Make sure you sign up for the correct
course number!

\end{frame}

%%%%%%%%%%%%%%%%%%%%%%%%%%%%%%%%%%%%%%%%%%%%%%%%%%%
\begin{frame}[fragile] \frametitle{}  \oldB \small

{\bf Problem Sets:}

There will be approximately $9$ problem sets assigned throughout the
semester, due on Thursdays. These will consist of both building custom
implementations of machine learning algorithms, as well as applying
established libraries to machine learning problems.
You should expect to become comfortable working simultaneously in a
number of programming languages. All submissions will be made
electronically on the ClassesV2 site.

\end{frame}

%%%%%%%%%%%%%%%%%%%%%%%%%%%%%%%%%%%%%%%%%%%%%%%%%%%
\begin{frame}[fragile] \frametitle{}  \oldB \small

{\bf Grading:}

Course grades will be determined based on scores from the problem
sets. I want to make the grading extremely transparent, so these
will all be graded on an 10 point scale (with the possibility of
up to one additional point for truly exceptional work or extra
credit questions). The final grade will be calculated by dropping
the lowest grade, rounding the average of remainder to the nearest
integer and reading off of the following table:

\end{frame}

%%%%%%%%%%%%%%%%%%%%%%%%%%%%%%%%%%%%%%%%%%%%%%%%%%%
\begin{frame}[fragile] \frametitle{} \oldB \small

{\bf Grading scale}

\begin{center}
\begin{tabular}{c || l | c}
Numeric Score & \multicolumn{2}{| l}{Final Grade} \\
\hline \hline
10 & A  & H \\
9  & A- & H \\
8  & B+ & HP \\
7  & B  & HP \\
6  & B- & HP \\
5  & C+ & P \\
4  & C  & P \\
3  & C- & P \\
2  & D  & F \\
1  & F  & F \\
0  & F  & F
\end{tabular}
\end{center}

\end{frame}

%%%%%%%%%%%%%%%%%%%%%%%%%%%%%%%%%%%%%%%%%%%%%%%%%%%
\begin{frame}[fragile] \frametitle{} \oldB \small

{\bf About me}

Joint appointment at Yale Statistics and AT\&T Labs Research
\begin{itemize}
\item Research focus on large-scale data analysis (think, many petabytes)
\item One focus is on encoding sparsity through penalized estimation
\item Applications to humanities and social sciences through with
analysis of image, text, and video corpora
\end{itemize}

\end{frame}

%%%%%%%%%%%%%%%%%%%%%%%%%%%%%%%%%%%%%%%%%%%%%%%%%%%
\begin{frame}[fragile] \frametitle{} \oldB \small

{\bf About you}

\begin{enumerate}
\item Name
\item Undergraduate/Graduatate; Major/Department/School; Year in Program
\item Prior stats and computer science courses taken at Yale
\item Do you work with R, Python, or both?
\item Why are you looking to get out of the course?
\item Any applications you are particularly excited about?
\end{enumerate}

\end{frame}


\end{document}













